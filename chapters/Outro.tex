\chapter{Outlook}\label{chapter:outlook}

In this work, we have introduced a novel method for fitting ensembles of neutrino flare candidates at a particular source location. This method was applied to two catalogs of source candidates motivated by TXS 0506+056: a catalog of northern-sky 3LAC blazars, and a "self-triggered" catalog of high energy IceCube events. Though neither catalog displayed a significant excess of neutrino flares, this allowed us to further constrain potential transient neutrino emission from blazars. 

The multi-flare algorithm introduced here was also applied on an all-sky scale. No significant population of neutrino flares was observed, though flare curves were produced at every location in the sky, providing a potentially useful tool for future multi-messenger searches wishing to incorporate information on the historical neutrino emission from source candidate locations, similar to what was done in~\cite{TXS_Archival}. 

The non-detection of a significant population of transient neutrino sources is somewhat interesting from the perspective of constraining astrophysical neutrino source populations. Whatever the sources of astrophysical neutrinos may be, they cannot produce neutrino flares that are so bright or numerous that the data becomes inconsistent with the background expectation on an all-sky scale. This places limits on the source density, flare rate, and neutrino burst energy that are not associated with any particular source class, serving as a "universal limit" of sorts on the behavior of potential astrophysical neutrino transients, as can be seen in figure~\ref{fig:mfskylim}. 

It should be noted, however, that a non-detection of neutrino sources on an all-sky scale is not entirely unexpected. The all-sky trial factor is quite large, and the most significant neutrino source candidates to date (TXS 0506+056 and NGC 1068) were both originally identified in spatially triggered analyses, where the candidate location was already identified as potentially interesting by other astrophysical messengers. Future, spatially triggered searches may be sensitive to individual sources that have been obscured by the all-sky trial factor. Ideally, the neutrino flare curves introduced in this work would aid with this, as the historical neutrino information provides an additional piece to the multi-messenger puzzle. 

Though the first applications of the multi-flare algorithm did not yield any significant discovery of astrophysical neutrino sources, the future of neutrino astronomy is still quite bright. With two interesting source candidates already identified, the answer to the question of the source of astrophysical neutrinos (and consequently, the source of high energy cosmic rays) is potentially beginning to become more clear. The upcoming years bring improvements to the IceCube detector calibration and event reconstruction, providing a more accurate description of existing data, as well as the construction of IceCube Gen2 which will boast increased effective area at high energies, thereby reducing the atmospheric neutrino background. Additionally, upcoming radio-based neutrino experiments such as RNO-G~\cite{RONG_design}, BEACON~\cite{Wissel_2020}, IceCube Gen2-Radio~\cite{Gen2paper}, PUEO~\cite{abarr2021payload}, and POEMMA~\cite{olinto2021poemma} will provide a view into the ultra-high energy neutrino regime, beyond what the current iteration of IceCube is realistically capable of measuring. As the identification of candidate neutrino emission from TXS 0506+056 was triggered by a single high energy neutrino event, the expansion of neutrino detectors into the high energy regime could lead the way to identifying future neutrino sources. 


