\chapter{Introduction}\label{chapter:introduction}

Though cosmic rays were first discovered in 1912, their origins have to date remained a mystery. Since cosmic rays are charged particles and their paths through the universe are bent by magnetic fields, identifying the sources of cosmic rays is not as simple as tracing back their arrival directions at Earth. The discovery of the astrophysical neutrino flux in 2013~\cite{astroneutrinos} provides a new avenue for investigating this problem, as the sources of astrophysical neutrinos are suspected to be the same as high energy cosmic rays, but neutrinos are not charged, and thus their arrival directions point directly back to their source. Identifying the sources of astrophysical neutrinos (and consequently cosmic rays) would provide an entirely new view of the universe, potentially allowing us to examine regions opaque to more traditional astrophysical messengers such as photons. 

With the IceCube Neutrino Observatory having collected over 10 years of data, the first hints of astrophysical neutrino sources are beginning to be visible: a high energy alert event from the direction of the blazar TXS 0506+056 sparked multi-messenger followup that suggests neutrino emission from that location~\cite{TXS_Multimessenger}\cite{TXS_Archival}, and recent studies of time integrated neutrino emission from candidate blazars reveal a $3.5 \sigma$ excess of astrophysical neutrino events from the direction of the Seyfert II galaxy NGC 1068~\cite{10yr_tint}. 

These first hints of astrophysical neutrino sources raise several further questions: Could there be more sources like TXS 0506+056 or NGC 1068? How many? Is neutrino emission time-dependent, or steady? Perhaps most importantly, if there are more sources, how can we identify them? Though neutrinos are an excellent astrophysical messenger due to the correlation of their arrival directions with their source of origin, neutrino astronomy is made significantly more difficult by the existence of the atmospheric neutrino flux, which produces a large, irreducible background for searches of astrophysical neutrino sources. Because of this, high energy neutrino astronomy relies either on triggering off a multi-messenger signal (such as a blazar that was also detected in gamma rays), or leveraging knowledge about the pattern of neutrino emission (for example, the arrival directions of astrophysical neutrinos should be clustered near their sources of origin, while the atmospheric background should be isotropically distributed). 

This work will focus specifically on exploring the possibility of time-dependent neutrino emission, motivated in part by the archival followup that was performed by IceCube at the location of the blazar TXS 0506+056~\cite{TXS_Archival}. A novel method for fitting decorrelated ensembles of neutrino flare candidates is introduced and subsequently applied to source catalogs assembled according properties of the TXS 0506+056 result that could potentially define a class of astrophysical neutrino sources. This new method of fitting ensembles of neutrino flares is intended to fill a methodological gap that exists in the field, as it provides a framework for combining ("stacking") neutrino flare candidates. 

This method is additionally applied to the entire neutrino sky, allowing for a more general examination of potential clustering of neutrino data in both space and time. If neutrino data is clustered beyond what is expected from the background hypothesis (isotropic arrival directions distributed uniformly in time), then that would be evidence of a population of astrophysical neutrino sources. Even in the case of a non-discovery, the method introduced here produces neutrino "light-curves" (here referred to as "flare curves") which are of potential use for future multi-messenger searches that may wish to incorporate information on historical neutrino emission from a particular source candidate.  

\begin{table}[h!]
\centering
 \begin{tabular}{||c c c||} 
 \hline
 . & Time-Integrated & Untriggered, Time-Dependent\\ [0.5ex] 
 \hline\hline
 Single Source & \cite{10yr_tint} & \cite{TXS_Archival} \\ 
 \hline
 Source Stacking & \cite{2lac_ic} & This work! \\
 \hline
\end{tabular}
\caption{Examples of some of the types of astrophysical neutrino source searches that have been performed using IceCube data. Time-integrated analyses search for an excess of events over the entire data sample livetime, ignoring any information about potential temporal clustering. By contrast, untriggered, time-dependent searches attempt to fit for temporal neutrino clusters ("flares") without using a multi-messenger lightcurve as a template. Single source searches report the most significant result from a small number of source candidate locations, corrected by a trial factor, while source stacking analyses combine information from many source locations under the assumption that each source candidate is a weak emitter, thereby increasing the search sensitivity to low individual source flux. }
\label{tab:stresults}
\end{table}



